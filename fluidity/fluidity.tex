% Created 2014-02-06 Thu 17:53
\documentclass[times,5pt]{article}
\usepackage[utf8]{inputenc}
\usepackage[T1]{fontenc}
\usepackage{fixltx2e}
\usepackage{graphicx}
\usepackage{longtable}
\usepackage{float}
\usepackage{wrapfig}
\usepackage{rotating}
\usepackage[normalem]{ulem}
\usepackage{amsmath}
\usepackage{textcomp}
\usepackage{marvosym}
\usepackage{wasysym}
\usepackage{amssymb}
\usepackage{hyperref}
\tolerance=1000
\usepackage[T1]{fontenc}
\usepackage{mathpazo}
\usepackage{tabularx}
\linespread{.95}
\usepackage[scaled]{helvet}
\usepackage{courier}
\usepackage{hyperref}
\hypersetup{
colorlinks,%
citecolor=black,%
filecolor=black,%
linkcolor=blue,%
urlcolor=black
}
\author{Fyodor Yarochkin, Vladimir Kropotov, Vitaliy Chetvertakov, Yennun Huang}
\date{\textit{<2014-01-25 Thu>}}
\title{The Age of Warring Nets: spotting malice in Fluid States of the Internet}
\hypersetup{
  pdfkeywords={},
  pdfsubject={Malice mining},
  pdfcreator={Emacs 24.3.50.1 (Org mode 8.2.5c)}}
\begin{document}

\maketitle

\section*{Abstract}
\label{sec-1}

The Internet as a global network is not a static system. It is ever
changing fluid platform with a number of dynamic characteristics. In
the early days of the Internet, the platform was designed as a
cooperative network of distributed machines, where malice or malicious
behavior was not expected, and could be easily resolved by isolating
the offending node. However, with the rapid growth of the global
network the situation changed dramatically. In the recent light of
discovery of mass state-sponsored surveillance, militarization of
Internet and global criminal activities we want to explore anomalies
in a dynamic Internet characteristics. In order to do so we passively
recorded and analyzed behavior of several network components, namely:
Domain Name System (DNS), Routing and Advertising Networks.

This paper presents our methodology, discusses our approach on
detecting anomalies and reveals findings.
\section*{Introduction}
\label{sec-2}

It is essential to understand the nature of modern Internet, which
resembles period of Warring States in Ancient China. State-sponsored
surveillance is wide-spread, criminal activities, raging from
targeted exploitation to mass criminal campaigns are common in modern
Internet.

In this study we monitor several key components of a modern Internet
System that affects nearly every network user - namely Domain Name
System, Routing and Advertising/Banner networks. We attempt to record
and document fluidity and volatility of each component and apply
analytics to identify anomalous and suspicious behavior within these
components, which could not be attributed to benign functionality of
the network.
\section*{Fluid States and Protocols}
\label{sec-3}
\subsection*{Passive DNS}
\label{sec-3-1}
\subsection*{Passive BGP}
\label{sec-3-2}
BGP is one of critical components of Internet Routing. Our motivation
of analyzing BGP routing fluidity was the fact known
attacks on BGP\cite{youtubehijack} that led to large-scale network
interception and network surveillance possibilities. We have
implemented a \textbf{passive BGP} monitoring platform that we discuss
further in this paper.
\subsection*{Passive HTTP and Ad networks}
\label{sec-3-3}
\section*{Anomaly Detection}
\label{sec-4}

\section*{Findings}
\label{sec-5}

\section*{Conclusions}
\label{sec-6}




\bibliographystyle{plain}
\bibliography{refs}{}
% Emacs 24.3.50.1 (Org mode 8.2.5c)
\end{document}